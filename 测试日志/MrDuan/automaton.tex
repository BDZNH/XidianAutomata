%!TEX program = xelatex
% \documentclass[tikz,convert=pdf2svg,preview]{standalone}
% \usepackage{amsmath}
% \usepackage{tikz}

% \usetikzlibrary{shapes,backgrounds}
% \usetikzlibrary{arrows, decorations.pathmorphing, backgrounds, positioning, fit, petri, automata} %自动机

% \usetikzlibrary{matrix,arrows}

% \usepackage{stmaryrd}  % \talloblong

% \begin{document}

% %% 第二章标记法结果
\begin{tikzpicture}[>=latex, shorten >=2pt,node distance=1in, on grid, auto]
    \node[state,initial] (q0) {$q_0$};
	\node[state] (q2) [right=of q0] {$q_2$};
	\node[state,accepting] (q4) [right=of q2] {$q_4$};
	\node[state] (q1) [above left=of q2] {$q_1$};
	\node[state,accepting] (q3) [above right=of q2] {$q_3$};
    \path[->]
	(q0) edge [below,bend right] node {$0$} (q4)
	(q2) edge [below] node {$0$} (q4)
	(q0) edge [below] node {$1$} (q2)
	(q1) edge [below] node {$0,1$} (q2)
	(q3) edge node {$0$} (q2)
	(q2) edge [loop above] node {$1$} (q2)
	(q4) edge [loop right] node {$0,1$} (q4)
	(q3) edge [loop right] node {$1$} (q3)
	;
\end{tikzpicture}

% \end{document}