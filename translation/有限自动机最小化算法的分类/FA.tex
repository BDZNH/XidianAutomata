\section*{B 有限自动机(FA, Finite automata)}
\addcontentsline{toc}{section}{B 有限自动机(FA, Finite automata)}

本节中我们定义有限自动机、其性质及其一些变化。大部分定义直接取自 \cite{Wats93}。
\newline

\noindent{\textbf{定义 B.1(有限自动机($FA$))}: 自动机是一个$6$元组$(Q,V,T,E,S,F)$,其中

\begin{itemize}
    \item[·] $Q$ 是有限状态集;
    \item[·] $V$ 是一个字母表;
    \item[·] $ T \in \mathcal{P}(P\times V \times Q) $是一个转换关系;
    \item[·] $ E \in \mathcal{P}(Q\times Q)$ 是一个 $\epsilon$-转换关系(空转换);
    \item[·] $ S \subseteq Q $是开始状态集;
    \item[·] $ F \subseteq Q $是结束状态集;     
\end{itemize}
字母表和函数$\mathcal{P}$的定义分别在“定义 A.8” 和 “惯例 A.1”。
\newline

\noindent{\textbf{备注 B.2}: 我们也会在状态转换关系的表示上采取一定的自由。例如,我们也把转移关系写成 $T\in V \longrightarrow \mathcal{P}(Q\times Q),T\in Q \times Q \longrightarrow \mathcal{P}(V),T\in Q \times V \longrightarrow \mathcal{P}(Q),T\in Q \longrightarrow \mathcal{P}(V\times Q),E\in Q \longrightarrow \mathcal{P}(Q)$。每种情况下,$Q$的从左到右的顺序会是“preserved”;例如,函数$T\in Q \longrightarrow \mathcal{P}(V \times Q)$ 定义为 $T(p)=\{ (a,q) : (p,a,q) \in T \}$。所使用的签名将从上下文中清除。详见备注 A.3。“$\longrightarrow$ ”的定义出现在惯例 A.2。

由于本文中我们只考虑有限自动机,所以我们将会频繁的使用简化术语“自动机”。

%%%%%%%%%%%%%%%%%%%%%%%%%%%%%%%%%%%%%%%%%%%%%%%%%%%%%%%%%%%%%
\subsection*{B.1 $FA$的性质}
\addcontentsline{toc}{subsection}{B.1 有限自动机的性质}
本小节将会定义一些有限自动机的性质。为了使定义更加简洁明了,我们引进三个特殊的$FA$: $M=(Q,V,T,E,S,F)$,$M_0=(Q_0,V_0,T_0,E_0,S_0,F_0)$,$ M_1=(Q_1,V_1,T_1,E_1,S_1,F_1) $。

\noindent{\textbf{定义 B.3(FA的大小)}:定义一个FA的大小为$|M|=|Q|$}。
\newline

\noindent{\textbf{定义 B.4(FA的同构$\cong$)}:我们把同构定义为FA的等价关系}。当且仅当 $V_0=V_1$,并且存在双射$g\in Q_0 \longrightarrow Q_1$ 

\begin{itemize}
    \item[·] $T_1 = \{ (g(p,q),a,g(q)) : (p,a,q) \in T_0 \}$
    \item[·] $E_1 = \{ (g(p,q),a,g(q)) : (p,q) \in E_0\}$
    \item[·] $S_1 = \{ g(s):s\in S_0 \}$
    \item[·] $F_1 = \{ g(f):f\in F_0 \}$
\end{itemize}
时$M_0$和$M_1$是同构的(写作$M_0 \cong M_1$)。
\newline

\noindent{\textbf{定义 B.5(转换关系$T$的扩展)}: 我们把$T \in V \longrightarrow \mathcal{P} (Q \times Q) $ 到 $ T^* \in V^* \longrightarrow \mathcal{P} (Q \times Q)  $的转换关系以如下方式扩展: } \\
\mbox{  } $T^*(\epsilon) = E^*$ \\
\mbox{且对于 } $(a\in V,w\in V^*)$ \mbox{ 有 }\\
\mbox{  } $ T^*(aw) = E^* \circ T(a) \circ T^*(w) $ \\
操作符$\circ$在惯例 A.5 中定义。\uline{这个定义也可以对称的表示}。
\newline

\noindent{\textbf{备注 B.6}:有时候我们也使用把转移关系写成: $T^* \in Q \times Q \longrightarrow \mathcal{P}(V^*)$}
\newline

\noindent{\textbf{定义 B.7(左语言和右语言)}:状态($M$中)的左语言由函数$ \overleftarrow{\mathcal{L}} _M \in Q \longrightarrow \mathcal{P}(V^*)$ 给出,其中:} \\
\mbox{  } $ \overleftarrow{\mathcal{L}}_M (q) = ( \cup s:s \in S : T^*(s,q) ) $ \\
状态($M$中)的右语言由函数$ \overrightarrow{\mathcal{L}} _M \in Q \longrightarrow \mathcal{P}(V^*)$给出,其中 \\
\mbox{  } $ \overrightarrow{\mathcal{L}}_M (q) = ( \cup f:f \in F : T^*(q,f) ) $ \\
通常在没有歧义的时候移除下标$M$。
\newline

\noindent{\textbf{定义 B.8(FA的语言)}:有限自动机的语言由函数 $\mathcal{L}_{FA} \in FA \longrightarrow \mathcal{P}(V^*) $给出,该函数的定义为:} \\
\mbox{  } $ \mathcal{L}_{FA} (M) = (\cup s,f:s \in S \land f \in F : T^* (s,f)) $ \\

\noindent{\textbf{定义 B.9(完全自动机($Complete$))}: 一个完全有限自动机满足 : } \\
\mbox{  } $ Complete(M) \equiv ( \forall q,a:q\in Q \land a \in V : T(q,a) \not= \emptyset )$ \\

\noindent{\textbf{定义 B.10 ($\epsilon$-$free$)}: 当且仅当 $E=\emptyset$时,$M$ 是 $\epsilon$-$free$ 的。
\newline

\noindent{\textbf{定义 B.11(Start-useful 自动机)}:一个 $Useful_s$ 有限自动机定义如下:} \\
\mbox{  } $ Useful_s (M) \equiv ( \forall q:q \in Q : \overleftarrow{\mathcal{L}} (q) \not= \emptyset ) $ \\

\noindent{\textbf{定义 B.12(Final-useful 自动机)}:一个 $Useful_f$ 有限自动机定义如下:} \\
\mbox{  } $ Useful_f (M) \equiv ( \forall q:q \in Q : \overrightarrow{\mathcal{L}} (q) \not= \emptyset ) $ \\

\noindent{\textbf{备注 B.13}: $Useful_s$和$Useful_f$与 $FA$ 的反转密切相关(见变换 B.22),对所有的 $M \in FA$,有 $Useful_f (M) \equiv Useful_s (M^R)$。 \\

\noindent{\textbf{定义 B.14(Useful 自动机)}:$Useful$ 有限自动机是一个只有可达状态的有限自动机:}\\
\mbox{  } $ Useful (M) \equiv Useful_s (M) \land Useful_f (M) $ \\

\noindent{\textbf{性质 B.15(确定性有限自动机($DFA$))}:当且仅当 }
\begin{itemize}
    \item [·] 无多重初始状态;
    \item [·] 无$\epsilon$转移;
    \item [·] 转移函数$T \in Q \times V \longrightarrow \mathcal{P} (Q) $ 不将 $Q \times V$ 映射至多重状态。
\end{itemize}
时有限自动机$M$是确定性的。\\
形式表达为:
$$ Det(M) \equiv ( |S| \leq 1 \land \epsilon-free(E) \land ( \forall q,a:q \in Q \land a \in V : |T(q,a)| \leq 1 )) $$

\noindent{\textbf{定义 B.16($FA$的确定性)}:$DFA$代表所有确定性的有限自动机的集合。我们把$FA \setminus DFA$称为非确定性有限自动机($NDFA$,$nondeterministic\ finite\ automata$)的集合。} \\

\noindent{\textbf{惯例 B.17(DFA的转换函数)}:对于$(Q,V,T,\emptyset,S,F)\in DFA$,我们考虑把转换函数记为$T\in Q \times V \not\rightarrow Q$($\not\rightarrow$的定义可以查看惯例A.2)。当且仅当 $DFA$ 是完全自动机的时候,转换函数是全函数。} \\

\noindent{\textbf{性质 B.18(弱确定性自动机)}:一些作者用比$Det$弱的确定性自动机的定义;使用左语言,定义如下:}
$$ Det'(M) \equiv (\forall q_0,q_1 : q_0 \in Q \land q_1 \in Q \land q_0 \not= q_1 : \overleftarrow{\mathcal{L}}(q_0) \cap \overleftarrow{\mathcal{L}}(q_1) = \emptyset ) $$
很容易证明$Det(M) \Rightarrow Det'(M)$。
\newline

\noindent{\textbf{定义 B.19($DFA$的最小化)}:满足以下条件时,$M\in DFA$是最小化的:}
$$ Min(M) \equiv (\forall M' : M' \in DFA \land Complete(M') \land \mathcal{L}(M) = \mathcal{L}_{FA}(M') : |M| \leq |M'| ) $$
$Min$仅定义在$DFA$上。如果我们定义一个最小的但是仍然完全的$DFA$,那么一些定义将会更加简单。它的定义如下:
$$ Min_{\mathcal{C}}(M) \equiv ( \forall M':M' \in DFA \land Complete(M') \land \mathcal{L}_{FA}(M) = \mathcal{L}_{FA}(M'): |M| \leq |M'| ) $$
$Min_{\mathcal{C}}$仅定义在完全$DFA$上。 \\

\noindent{\textbf{性质 B.20($DFA$的最小化)}:根据 Myhill-Nerode 定理,\underline{An M, such that Min(M)},是唯一的最小化$DFA$,定理的相关介绍在 \cite{Hu79,Wats93} }。 \\

\noindent{\textbf{性质 B.21($DFA$最小化的一个替代定义)}:为了最小化$DFA$,使用定义(仅定义在$DFA$上):} \\
\mbox{  }$Minimal(Q,V,T,\emptyset,S,F) \equiv  ( \forall q_0,q_1:q_0 \in Q \land q_1 \in Q \land q_0 \not= q_1 : \overrightarrow{\mathcal{L}}(q_0) \not= \overrightarrow{\mathcal{L}}(q_1)  )$ \\
\mbox{              } $\land Useful(Q,V,T,\emptyset,S,F)$ \\
有 $Minimal(M) \equiv Min(M)$(对所有$M\in DFA$)。很容易证明$Min(M) \Rightarrow Minimal(M)$。\uline{The reverse direction follows from the Myhill-Nerode 定理}。

与$Min_{\mathcal{C}}$相似的定义是(同样也只定义在$DFA$上):\\
\mbox{  }$Minimal_{\mathcal{C}}(Q,V,T,\emptyset,S,F) \equiv  ( \forall q_0,q_1:q_0 \in Q \land q_1 \in Q \land q_0 \not= q_1 : \overrightarrow{\mathcal{L}}(q_0) \not= \overrightarrow{\mathcal{L}}(q_1)  )$ \\
\mbox{              } $\land Useful_s (Q,V,T,\emptyset,S,F)$ \\
有 $Minimal_{\mathcal{C}}(M) \equiv Min_{\mathcal{C}}(M)$的性质(对于所有的$M\in DFA$)。很容易证明$Min_{\mathcal{C}}(M) \Rightarrow Minimal_{\mathcal{C}}(M)$。\uline{The reverse direction follows from the Myhill-Nerode 定理}。

%%%%%%%%%%%%%%%%%%%%%%%%%%%%%%%%%%%%%%%%%%%%%%%%%%%%%%%%%%%%%%%%%%%%
\subsection*{B.2 有限自动机的变换}
\addcontentsline{toc}{subsection}{B.2 有限自动机的变换}

\noindent{\textbf{变换 B.22($FA$ 反转)}:$FA$反转由后缀(上标)函数$ R \in FA \longrightarrow FA $ 给出,它的定义如下:}
$$ (Q,V,T,S,F)^R = (Q,V,T^R,E^R,F,S) $$ 
函数 $R$ 满足
$$(\forall M : M \in FA : ( \mathcal{L} (M) )^R = \mathcal{L}_{FA}(M^R)) $$
\newline

\noindent{\textbf{变换 B.23(移除开始状态不可达状态)}:变换$useful_s \in FA \longrightarrow FA$移除开始状态不可达状态: }\\
\mbox{   }$useful_s(Q,V,T,E,S,F) = $ \mbox{\textbf{let  }} $U = SReachable(Q,V,T,E,S,F)$ \\
\mbox{               }\mbox{\textbf{ in }} \\
\mbox{                   } $ (U,V,T \cap (U\times V \times U), E \cap (U \times U), S \cap U, F \cap U ) $ \\
\mbox{               }\mbox{\textbf{ end }} \\
函数 $ useful_s $满足
$$ (\forall M : M \in FA : Useful_s ( useful_s(M) ) \land \mathcal{L}_{FA} (useful_s(M)) = \mathcal{L}_{FA}(M)) $$

\noindent{\textbf{变换 B.24(子集构造)}:函数$subset$把一个$\epsilon$-$free$ $FA$ 转换为一个 $DFA$ (in the \textbf{let} clause $T'\in \mathcal{P}(Q) \times V \longrightarrow \mathcal{P}(\mathcal{P} (Q) )$)}: \\
\mbox{  } $subset(Q,V,T,\emptyset,S,F) = \mbox{\textbf{let  }} T'(U,a) = \{ (q:q\in U : T(q,a) ) \} $\\
\mbox{                }$F'= \{ U : U \in \mathcal{P}(Q) \land U \cap F \not= \emptyset \} $ \\
\mbox{             \textbf{in}} \\
\mbox{                }$ ( \mathcal{P}(Q),V,T',\emptyset,\{ S \},F' ) $ \\
\mbox{             \textbf{end}} \\
除了明显的性质$\mathcal{L}_{FA}(subset(M)) = \mathcal{L}_{FA}(M)$之外(对所有的$M\in FA$),函数$subset$满足:
$$ (\forall M:M \in FA \land \epsilon\mbox{-}free(M): Det(subset(M)) \land Complete( subset(M ))) $$
它也被认为是“幂集”构造。
\newline

\noindent{\textbf{性质 B.25(子集构造)}:设 $M_0=( Q_0,v,t_0,\emptyset,S_0,F_0 )$和 $M_1 = subset(M_0)$为有限自动机。通过子集构造,状态集$M_1$成为$\mathcal{P}(Q_0)$。有如下性质:  }
$$ (\forall p:p \in \mathcal{P}(Q_0) : \overrightarrow{\mathcal{L}}_{M_1}(p) = ( q:q \in p : \overrightarrow{\mathcal{L}}_{M_1}(q) ) ) $$

\noindent{\textbf{定义 B.26(优化子集构造)}:函数$subsetopt$把一个$\epsilon$-$free$ $FA$ 转换为一个 $DFA$ 。此函数是$subset$的一个优化版本:  } \\
\mbox{  } $subsetopt(Q,V,T,\emptyset,S,F) = \mbox{\textbf{ let   }} T'(U,a) = \{ (q:q\in U : T(q,a) ) \} $\\
\mbox{                   }$ Q' = \mathcal{P} (Q) \setminus \{ \emptyset \} $ \\
\mbox{                   }$F'= \{ U : U \in \mathcal{P}(Q) \land U \cap F \not= \emptyset \} $ \\
\mbox{               \textbf{in}} \\
\mbox{                   }$ ( \mathcal{P}(Q),V,T',\emptyset,\{ S \},F' ) $ \\
\mbox{               \textbf{end}} \\
除了性质$\mathcal{L}_{FA} (subsetopt(M)) = \mathcal{{L}} (M) $(对所有的$M\in FA$)之外,函数$subsetopt$还满足
$$ ( \forall M : M \in FA \land \epsilon \mbox{-}free(M) : Det(subset(M)) )  $$

\newpage

