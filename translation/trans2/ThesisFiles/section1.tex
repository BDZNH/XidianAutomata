
%%正文页
% \newpage
% \setcounter{page}{1}

\chapter{介绍}

自1950年来,确定性有限自动机的最小化就一直在研究当中。简单来说就是找到一个唯一的(直至同构)最小的确定性有限自动机,它能接收与给定的确定性有限自动机相同的语言。解决这一问题的算法应用范围很广,从编译器构造到硬件电路的最小化都有它的身影。有了形式多样的应用程序,不同的表示形式的数量也在增加:大多数教科书都有自己的变体,而时间复杂度最优的算法 (Hopcroft的算法) 仍然晦涩难懂。

本文介绍了有限自动机最小化算法的有关分类。如下所示:

\begin{itemize}
    \item 大多数教材的作者称他们的最小化算法由 Huffman 算法 \cite{Huff54} 和 Moore 算法 \cite{Moor56} 直接推导得到。不幸的是,大多数教材都展示了截然不同的算法 (比如 \cite{AU92,ASU86,Hu79,Wood87}), 只有由 Aho 和 Ullman 发表的算法直接源自 \cite{Huff54, Moor56}。
    \item 虽然大多数算法依赖于计算状态的等价关系,但伴随算法演示的许多解释并未明确提及算法是计算等价关系、它包含的状态划分还是它的补充。
    \item 算法之间的比较进一步受到呈现方式的巨大差异的阻碍——有时用命令式程序展示、有时使用函数式程序,但通常只作为描述性段落。
\end{itemize}

有限自动机构造算法的相关分类在\cite{Wats93}中。

除了其中一个算法(Brzozowski的算法)之外,其他所有算法依赖于确定等价\footnote{状态的等价稍后定义。}的自动机状态的集合。在第 2 章中讨论了不使用等价状态的算法。在第 3 章中给出了状态等价的定义和一些性质。计算等价状态的算法在第 4 章中给出。分类的主要结果被总结在结论部分 5 中。附录 A 和附录 B 给出了阅读本文所需的基本定义。与有限自动机相关的定义取自\cite{Wats93}。图 \ref{fig:1} 展示了最小化算法之间的关系。

大多数最小化算法的主要计算在于确定等价的(或不等价的)状态,从而在状态上产生等价关系。在本文中,我们考虑以下最小化算法:
\begin{itemize}
    \item Brzozowski(可能是非确定性的)有限自动机最小化算法 在 \cite{Brzo62} 中提出。这个优雅的算法(第2节)最初是由Brzozowski提出,此后又在没有Brzozowski的“参与”的情况下被重新提出。给出了一个不含$\epsilon$-转移的有限自动机(可能不确定的),该算法生成一个最小的接受相同的语言的确定性有限自动机。

    \item 分层等价计算等价于 \cite{Wood87, Moor56, Brau88, Urba89} 中提出。算法(算法 4.2)是由状态等价的定点定义产生的近似序列所建议的直接实现。

    \item 等价性的无序计算。该算法(算法4.3,未出现在文献中)计算等价关系;以任意顺序选择状态对(考虑等价性)。

    \item 等价类的无序计算发表在\cite{ASU86}。该算法(算法4.4)是上述计算状态等价性的算法的一种更改。

\end{itemize}

\begin{figure}[tbp]
    \centering
    \resizebox {0.8\textwidth} {!} {
        \begin{tikzpicture}
            [place/.style={circle,draw=black,fill=black,thick}]
            % \node at ( 0, 0) [place,label=right:$\text{状态的等价 (}\S 3)$] (n0) {};
            % \node at ( 6,-4) [place,label=right:$(\S \text{4.6})$] (n1) {};
            % \node at (-6,-4) [place,label=right:$(\S \text{ 4.1-4.5,4.7})$] (n2) {};
            % \node at ( 6,-10) [place,label=right:$(\text{4.9})$] (n3) {};
            % \node at ( 6,-14) [place,label=right:$(\text{pg. 15})$] (n4) {};
            % \node at ( 0,-8) [place,label=right:$(\text{4.10})$] (n15) {};
            % \node at (-12,-8) [place,label=right:$(\S \text{4.1-4.5})$] (n5) {};
            % \node at (-6,-14) [place,label=right:$(\text{Hopcroft-Ullman(4.7)})$] (n6) {};
            % \node at (-18,-14) [place,label=right:$(\text{4.2})$] (n7) {};
            % \node at (-12,-14) [place] (n8) {};
            % \node at (-16,-18) [place,label=right:$(\text{4.3})$] (n9) {};
            % \node at (-16,-22) [place,label=right:$(\text{ASU(4.4)})$] (n10) {};
            % \node at (-8,-18) [place,label=right:$(\text{4.5})$] (n11) {};
            % \node at (-8,-22) [place,label=right:$(\text{4.6})$] (n12) {};
            % \node at (-8,-26) [place,label=right:$(\text{pg. 14})$] (n13) {};
            % \node at (-8,-30) [place,label=right:$(\text{Hopcroft(4.8)})$] (n14) {};
            % \node at (-18,-0) [place,label=right:$\text{Brzozowski}(\S 2)$] (n16) {};

            \node at ( 0, 0) [place,label=right:$\text{状态的等价 (}\S 3)$] (n0) {};
            \node at ( 6,-3) [place,label=right:$(\S \text{4.6})$] (n1) {};
            \node at (-6,-3) [place,label=right:$(\S \text{ 4.1-4.5,4.7})$] (n2) {};
            \node at ( 6,-8) [place,label=right:$(\text{4.9})$] (n3) {};
            \node at ( 6,-12) [place,label=right:$(\text{pg. 15})$] (n4) {};
            \node at ( 0,-7) [place,label=right:$(\text{4.10})$] (n15) {};
            \node at (-12,-7) [place,label=right:$(\S \text{4.1-4.5})$] (n5) {};
            \node at (-6,-12) [place,label=right:$(\text{Hopcroft-Ullman(4.7)})$] (n6) {};
            \node at (-18,-12) [place,label=right:$(\text{4.2})$] (n7) {};
            \node at (-12,-12) [place] (n8) {};
            \node at (-16,-15) [place,label=right:$(\text{4.3})$] (n9) {};
            \node at (-16,-18) [place,label=right:$(\text{ASU(4.4)})$] (n10) {};
            \node at (-8,-15) [place,label=right:$(\text{4.5})$] (n11) {};
            \node at (-8,-18) [place,label=right:$(\text{4.6})$] (n12) {};
            \node at (-8,-21) [place,label=right:$(\text{pg. 14})$] (n13) {};
            \node at (-8,-25) [place,label=right:$(\text{Hopcroft(4.8)})$] (n14) {};
            \node at (-18,-0) [place,label=right:$\text{Brzozowski}(\S 2)$] (n16) {};
          
            \draw[-] (n0) to [right] node {逐点} (n1);
            \draw[-] (n1) to [right] node {命令式程序} (n3);
            \draw[-] (n3) to [right] node {记忆化} (n4);
            \draw[-] (n0) to [left] node {等价关系} (n2);
            \draw[-] (n2) to [right] node {从下面逼近} (n15);
            \draw[-] (n2) to [left] node {从上面逼近} (n5);
            \draw[-] (n5) to [right] node {状态对} (n6);
            \draw[-] (n5) to [right] node {分层} (n7);
            \draw[-] (n5) to [right] node {无序} (n8);
            \draw[-] (n8) to [right] node {原生} (n9);
            \draw[-] (n9) to [right] node {等价类} (n10);
            \draw[-] (n8) to [right] node {提升} (n11);
            \draw[-] (n11) to [right] node {等价类} (n12);
            \draw[-] (n12) to [right] node {列表} (n13);
            \draw[-] (n13) to [right] node {优化列表更新} (n14);
          
          \end{tikzpicture}
    }
    \caption{}
    \label{fig:1}
\end{figure}
% \clearpage
图 \ref{fig:1} :有限自动机最小化算法之间的关系。Brzozowski的最小化算法与其他算法无关,并作为一个单独的顶点出现。本文提出的每一个算法都作为树的一个顶点出现。对于本文中明确展示的每个算法,则在括号中标示出它在本文中出现的位置。对于没有明确展示的算法,给出了相应的页码。边表示解的细化(也即算法之间的关系)。边上面标记了细化的名称。

\begin{itemize}

    \item 改进的等价的无序计算。这个算法(算法4.5,没有出现在文献中)也以任意顺序计算等价关系。该算法是对其他无序算法的一个小改进。

    \item 改进了类的无序计算。该算法(算法4.6,不在文献中)是上述用来计算状态的等价类的算法的修改。该算法用于Hopcroft最小化算法的推导。

    \item Hopcroft 和 Ullman 的算法在\cite{Hu79}中提出。该算法(算法4.7)计算不等价的(可区分的)关系。虽然它是基于 Huffinan \cite{Huff54}和 Moore \cite{Moor56} 的算法,但该算法使用一些有趣的编码技术。

    \item Hopcroft 的算法在 \cite{Hopc71, Grie73} 提出 。该算法(算法4.8)是用于最小化的最有名的算法(在运行时间分析方面)。由于 Hopcroft 的原始陈述是难以理解的,所以本文的对该算法介绍基于Gries 的文章。

    \item 等价的点态计算。该算法(算法4.9,不在文献中)计算给定状态对的等价性。它借鉴了一些非自动机相关的技术,例如:类型的结构等价和函数式程序的记忆化。

    \item \uline{Computation of equivalence from below (with respect to refinement). }
    \uline{从下面(相对于细化)计算等价性。该算法(算法4.10,不在文献中)计算从下面的等价关系。}与任何其他已知算法不同,该算法的中间结果可用于构造较小的(虽然不是最小的)确定性有限自动机。

\end{itemize}