\chapter{总结}

\noindent{最小化算法的总结如下:}

\begin{itemize}
    \item 给出了Brzozowski最小化算法的推导。这个推导比原始推导(Brzozowski)更容易理解,或者 van de Snepscheut 给出的推导更容易理解。给出了最小化算法的简史。希望解决其发现的一些错误归因。
    \item 等价性(关系$E$)和可区分性(关系$D$)作为某些方程的固定点的定义比许多教科书演示更容易理解。
    \item $E$的不动点特征使得计算$E$(或$D$)所需的逼近步骤的数目的上界特别容易。这个上界后来被证明在确定一些算法的时间复杂度和逐点算法效率提升的上是有用的。
    \item 作为最大不动点的$E$的定义有助于识别所有(先前)已知的算法从上面计算出$E$(关于细化)。因此,这些算法全都具有在最小化有限自动机的过程中不可用的中间结果。
    \item 在相同的框架下,我们成功地给出了所有已知的text-book算法。他们中的大多数被证明本质上是相同的,仅在它们的环结构上有微小的差异。一个例外是 Hopcroft 和Ullman 的算法 \cite{Hu79},它具有明显不同的循环结构。本文提出的算法(带有不变量),比原来的表述更容易理解。我们的报告突出了这样一个事实,即算法中的主要数据结构不必是列表,一个集合就足够了。 
    \item Hopcroft 的最小化算法\cite{Hopc71}最初是以一种不太容易理解的方式呈现的。就像Gries的文章\cite{Grie73}。我们努力以清晰和精确的方式推导出该算法。本文中的介绍突出了两个重点:该算法的推导起点是一种易于理解的简单算法;,并且在 Hopcroft 和 Gries 的这种算法的呈现中使用列表数据结构是不必要的——可以使用集合。
    \item  本文提出了一些新的最小化算法,其中许多是已知算法的变体。有两个新算法(在第4.6和4.7节中给出)不是推导自任何已知算法,--并且在它们自己的版权中是重要的--。
    \begin{itemize}
        \item 提出了一种以逐点法计算关系E的算法。该算法由一个用于确定结构等价类型的算法优化而来。在优化构成中某些技术扮演了重要角色:
        \begin{itemize}
            \item 计算$E$所需的步骤数的上界被用来通过限制逐点计算$E$时需要考虑的状态对的数量来改进算法;
            \item 算法的函数式程序部分的记忆化(memoization)用于减少冗余计算量。 
        \end{itemize} 
        \item 提出了一种新的计算$E$的算法。该算法利用$E$的逐点计算来构造和细化$E$的近似。因为计算是从下面,该算法的中间结果可用于(至少部分地)减小$DFA$的大小。这可用于$DFA$最小化时间量受限的应用中(如在实时应用中)。相反,所有(先前)已知的算法具有不可用的中间结果。
    \end{itemize} 
\end{itemize}

\newpage
