%!TEX root = ../Demo.tex
% \chapter*{A 一些基本定义}
% \addcontentsline{toc}{chapter}{A 一些基本定义}
\chapter{一些基本定义}

\begin{convention}[幂集] 
    对于任意集合$A$,我们使用$\mathcal{P}(A)$代表$A$的所有子集。$\mathcal{P}(A)$也叫做$A$的幂集。有时也写作 $2^A$。
\end{convention}

\begin{convention}[函数集] \label{pro:mathP}
    对于集合 $A$ 和 $B$ ,$A\to B$代表所有从$A$到$B$的函数的集合。而 $ A \not\to B$代表所有从$A$到$B$的“partial functions”。
\end{convention}


\begin{remark}
    对于集合$A,B$,关系 $\mathcal{C}\subseteq A \times B$ ,我们可以把$\mathcal{C}$理解为函数 $\mathcal{C} \in A \longrightarrow \mathcal{P}(B)$。
\end{remark}



\begin{convention}[元组投影]
    对于$n$元组 $t=(x_1,x_2,\cdots,x_n)$,我们使用符号${\pi}_i(t)$ $(1 \leq i \leq n)$ 代表元组元素$x_i$;我们使用符号 $\bar{\pi}_i(1 \leq i \leq n)$代表$(n\mbox{-}1)$元组$(x_1,\cdots,x_{i-1},x_{i+1},\cdots,x_n)$。$\pi$ 和 $\bar{\pi}$ 自然扩展到元组集。 
\end{convention}



\begin{convention}[组合关系]
    给出集合$A,B,C$和两个关系$E \subseteq A \times B$ 和 $ F \subseteq B \times C $,定义组合关系(插入操作符$\circ$):
$$ E\circ F = \{ (a,c) : (\exists b:b \in B : (a,b) \in E \land (b,c) \in F) \} $$ 
\end{convention}



\begin{convention}[等价关系的等价类]
    对任何集合 $A$ 上的等价关系 $E$,我们使用 $[A]_E$代表等价类集合,即:
$$ [A]_E = \{ [a]_E :a \in A \} $$
集合 $[A]_E$ 也叫做 $A$ 的由 $E$ 引出的“划分(partition)”。
\end{convention}


\begin{definition}[等价类的指数]
    对于集合$A$上的等价关系$E$,定义$\sharp E = | [A]_E |$。$\sharp E$ 也叫做$E$的“指数”。
\end{definition}


\begin{definition}[字母表] \label{def:Alphabat}
    字母表是有限大小的非空集合。
\end{definition}



\begin{definition}[等价关系的细化]
    对于等价关系$E$和$E'$(在集合$A$上),当且仅当$E \subseteq E'$,$E$是$E'$的“细化”。
\end{definition}

\begin{definition}[划分的细化关系$\sqsubseteq$]
    对于等价关系$E$和$E'$(在集合$A$上),当且仅当$ E \subseteq E' $, $[A]_E$也被称为$ [A]_{E'} $的细化(写作$ [A]_E \sqsubseteq [A]_{E'} $)。当且仅当 $E$ 下的每一个等价类完全包含在$E'$下的某些等价类时,等价命题是$ [A]_E \sqsubseteq [A]_{E'} $ 。
\end{definition}


\begin{definition}[元组和关系反转]
    对一个$n$元组$(x_1,x_2,\cdots,x_n)$,定义反转为函数$R$(后缀和上标)
    $$ (x_1,x_2,\cdots,x_n)^R = (x_n,\cdots,x_2,x_1) $$
给出一个集合元组$A$,定义$A^R = \{ x^R:x\in A \}$。
\end{definition}


