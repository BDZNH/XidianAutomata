\chapter{Brzozowski提出的算法}

\uline{In the case of a nondeterministic FA, the subset construction is applied first, followed by the minimization algorithm.} In this section, we consider the possibility of applying the subset construction (with useless state removal) after an (as yet unknown) algorithm to yield a minimal DFA. We now construct such an algorithm.

大多数最小化算法应用于 $DFA$ 。\uline{对于不确定性有限自动机,首先应用子集构造,然后应用最小化算法}。在本节中,我们将考虑在(未知的)算法之后应用子集构造(带无用状态删除)以生成最小$DFA$的可能性。我们现在构造这样的算法。(在本节中描述的算法也可用于通过用 $subset$ 替换函数 $subsetopt$ 来构造最小完全$DFA$)。

令$M_0 = (Q_0,V,T_0,\emptyset ,S_0,F_0)$ 为一个 $\epsilon$-$free$ $FA$,对其进行最小化,令 $M_2 = ( Q_2,V,T_2,\emptyset,S_2,F_2 ) $ 为最小的 $DFA$,那么 $ \mathcal{L} (M_0) = \mathcal{L}(M_2) $ ( $Min(M_2)$ 当然也是,详见定义 B.19)。(对本节中剩下的部分我们使用 $Minimal$(性质 B.21))。由于最后构造子集,有一些中间$FA$ $M_1 = ( Q_1,V,T_1,\emptyset,S_1,F_1 )$,于是$M_2 = useful \circ subsetopt(M_1)$。我们要求 $M_1$ 是由 $M_0$ 得到的,那么 $ \mathcal{L}_{FA}(M_2) = \mathcal{L}_{FA}(M_1) = \mathcal{L}_{FA}(M_0)$。

从 $Minimal(M_2)$  (性质 B.21)开始,我们需要:
$$ (\forall p,q,:p\in Q_2 \land q \in Q_2 \land p \not= q : \overrightarrow{\mathcal{L}}(p) \not= \overrightarrow{\mathcal{L}}(q)) \land useful(M_2)   $$ 
对与所有的状态$q\in Q_2$,由于 $M_2 = useful_s \circ subsetopt(M_1)$ ,所以有 $q\in P(Q_1$ 。性质 B.25 的子集构造给出: 
$$ (\forall p:p\in Q_2 : \overrightarrow{\mathcal{L}}(p) = (\cup q:q \in Q_1 \land q \in p:\overrightarrow{\mathcal{L}}(q)) ) $$ 
我们需要 $M_1$ 上充足的条件来确保 $Minimal(M_2)$。相应条件的推导如下: \\
\mbox{   } $Minimal(M_2)$ \\
\mbox{ } $\equiv$ \mbox{  } {$Minimal$的定义(性质 B.21)} \\
\mbox{   } $ (\forall p,q,:p\in Q_2 \land q \in Q_2 \land p \not= q : \overrightarrow{\mathcal{L}}(p) \not= \overrightarrow{\mathcal{L}}(q)) \land useful(M_2)$ \\
\mbox{ } $ \Leftarrow $ \mbox{  } { 性质 B.25; $M_2 = useful \circ subsetopt(M_1)$ } \\
\mbox{   }$ (\forall p,q,:p\in Q_2 \land q \in Q_2 \land p \not= q : \overrightarrow{\mathcal{L}}(p) \not= \overrightarrow{\mathcal{L}}(q) = \emptyset) \land useful(M_2)$ \\
\mbox{ } $\equiv $ \mbox{  } {$Det'$的定义(性质 B.18)和 $useful_s, Useful_f $(备注 B.13) } \\
\mbox{   } $Det'(M^R_1) \land Useful(M^R_1)$ \\
\mbox{ } $\Leftarrow $ \mbox{  } {$Det'(M) \Leftarrow Det(M)$} \\
\mbox{   } $Det(M^R_1) \land Useful(M^R_1)$ \\
$M_1$需要的条件可以通过 $M_1=R \circ useful_s \circ subsetopt \circ R(M_0)$ (反转前缀)建立。

完全最小化算法(对于任何 $\epsilon$-$free$ $M_0\in FA$ )如下:
$$  M_2 = useful_s \circ subsetopt \circ R \circ useful_s \circ subsetopt \circ R(M_0) $$

该算法最初由 Brzozowski 在 \cite{Brzo62} 中提出。最初 Jan van de Snepscheut 在他的博士论文 \cite{vdSn85}中提出该算法时是\uline{模糊的}。在该论文中,算法来源于一个教授的私人通讯——埃因霍芬理工大学的 Pereman, Pereman 最初在 Mirkin 的文章 \cite{Mirk65} 中找到该算法。虽然 Mirkin 引用了 Brzozowki \cite{Brzo64} 的论文,但 Mirkin 的作品是否受 Brzozowki 的最小化工作的影响尚不清楚。Jan van de Snepscheut 的新书\cite{VDSn93} 描述了该算法,但既不提供该算法历史,也不提供该算法的引文(除了他自己的论文外)。
